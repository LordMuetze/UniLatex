% !TeX root = 2021-10-18_Uebungsblatt1.tex



%%%%%%%%%%%%%%%%%%%%%%%%%%%%%%%%%%%%%%%%%%%%%%%%%%%%%%%%%%%%%%%%%%%%%%%%%%%%%%%%
\documentclass[12pt]{article}
\usepackage[a4paper, left=1.5cm, right=1.5cm, top=3.5cm, bottom=1.5cm]{geometry}


\usepackage[utf8]{inputenc} % Textkodierung: UTF-8
\usepackage[T1]{fontenc} % Zeichensatzkodierung

\usepackage[ngerman]{babel} % Deutsche Lokalisierung
\usepackage{graphicx} % Grafiken
\usepackage{calc} % Berechnungen
\usepackage{titling}
\usepackage{xcolor}

% Silbentrennung:
\usepackage{hyphenat}
%\tolerance 2414
%\hbadness 2414
%\emergencystretch 1.5em
%\hfuzz 0.3pt
%\widowpenalty=10000     % Hurenkinder
%\clubpenalty=10000      % Schusterjungen
%\vfuzz \hfuzz

% Euro-Symbol:
\usepackage[gen]{eurosym}
\DeclareUnicodeCharacter{20AC}{\euro{}}

\usepackage{calc}
% Schriftart Helvetica:
%\usepackage[scaled]{helvet}
\renewcommand{\familydefault}{\sfdefault}

\usepackage{bookmark} % Lesezeichen

\usepackage{amsmath}
\usepackage{mathptmx} % skalierbare Formelschriften

\usepackage{multicol} % mehrspaltiger Text
%\usepackage{showframe} % Seitenbegrenzungen anzeigen
\usepackage{lipsum} % Blindtext

% Unterdrückung layoutbedingter Warnungen
\usepackage[immediate]{silence}
\WarningFilter[layout]{Fancyhdr}{\footskip is too small (18.4941pt)} % Fußzeile A4 Hoch und Quer
\WarningFilter[layout]{Fancyhdr}{headheight is too small (39.83385pt)} % Kopfzeile A4 Quer

% Debugging:
%\DeactivateWarningFilters[layout] % Unterdrückte Warnungen einschalten



% Längen:
\newlength{\PlakatSeiteHoehe}
\newlength{\PlakatSeiteRand}
\newlength{\PlakatKopfzeileHoehe}
\newlength{\PlakatKopfzeileAbstand}
\newlength{\PlakatFusszeileHoehe}
\newlength{\PlakatUniversitaetLogoBreite}
\newlength{\PlakatFakultaetLogoBreite}
\newlength{\PlakatPositionLinks}
\newlength{\PlakatKopfzeilePositionUnten}
\newlength{\PlakatFakultaetslogoPositionKorrekturLinks}
\newlength{\PlakatFakultaetslogoPositionKorrekturOben}
\newlength{\PlakatFakultaetslogoAbstandRechts}
\newlength{\PlakatFusszeilePositionUnten}
\newlength{\PlakatFusszeilePositionKorrektur}
\newlength{\PlakatTitelZweiPlatzDanach}
\newlength{\PlakatTitelDreiPlatzDanach}
\newlength{\PlakatUniversitaetslogoPositionKorrekturLinks}
\newlength{\PlakatSpaltenAbstand}
\newlength{\PlakatSchriftNormalGroesse}
\newlength{\PlakatSchriftNormalZeilenabstand}
\newlength{\PlakatAufzaehlungAbstandLinks}


\setlength{\PlakatFusszeilePositionUnten}{-5.5mm}




\setlength{\PlakatSeiteRand}{1.5cm}
\setlength{\PlakatKopfzeileHoehe}{14mm}
\setlength{\PlakatKopfzeileAbstand}{0mm}
\setlength{\PlakatKopfzeilePositionUnten}{10mm}
\setlength{\PlakatUniversitaetLogoBreite}{19mm}
\setlength{\PlakatFakultaetLogoBreite}{12mm}
\setlength{\PlakatPositionLinks}{0mm}
\setlength{\PlakatFakultaetslogoPositionKorrekturLinks}{10mm}
\setlength{\PlakatFakultaetslogoPositionKorrekturOben}{0.7mm}
\setlength{\PlakatFakultaetslogoAbstandRechts}{0mm}
\setlength{\PlakatUniversitaetslogoPositionKorrekturLinks}{0mm}
\setlength{\PlakatSpaltenAbstand}{7mm}
\setlength{\PlakatSchriftNormalGroesse}{11pt}
\setlength{\PlakatSchriftNormalZeilenabstand}{15pt}
\setlength{\PlakatFusszeileHoehe}{15mm}
\setlength{\PlakatAufzaehlungAbstandLinks}{0.9em}


\newcommand{\PlakatKopfzeileSchriftgroesse}{\fontsize{9}{11}}
\newcommand{\PlakatSchriftStandard}{\PlakatSchriftNormalGroesse\selectfont}
\newcommand{\PlakatFusszeileSchrift}{\fontsize{9}{10.8}\selectfont}
\newcommand{\PlakatBildUnterschrift}[1]{{\\[1pt]\fontsize{10}{12}\selectfont{}#1\vspace{-\baselineskip}}}



% Definition of \maketitle
\makeatletter         
\def\@maketitle{
\raggedright
%\includegraphics[width=\PlakatFakultaetLogoBreite]{./Ressourcen/_Bilder/Fakultaet_Logo_RGB.pdf}%\\[8ex]
\begin{textblock*}{\PlakatFakultaetLogoBreite}%
    [1,1](\PlakatFakultaetslogoPositionKorrekturLinks,
        \PlakatKopfzeilePositionUnten + \PlakatFakultaetslogoPositionKorrekturOben)% 
\includegraphics[width=\PlakatFakultaetLogoBreite]{./Ressourcen/_Bilder/Fakultaet_Logo_RGB.pdf}%
\end{textblock*}%
\begin{textblock*}{\textwidth - \PlakatFakultaetLogoBreite - \PlakatFakultaetslogoAbstandRechts}%
    [0,1](\PlakatPositionLinks + \PlakatFakultaetLogoBreite + \PlakatFakultaetslogoAbstandRechts, \PlakatKopfzeilePositionUnten)%
\color{UniversitaetFarbe}%
\PlakatKopfzeileSchriftgroesse\selectfont%
\LehrVeranstaltung\\%
\theauthor\\%
\thetitle\\%
\end{textblock*}%
\begin{textblock*}{\PlakatUniversitaetLogoBreite}[1,1]%
    (\paperwidth - \PlakatSeiteRand - \PlakatSeiteRand + \PlakatUniversitaetslogoPositionKorrekturLinks, \PlakatKopfzeilePositionUnten)%
\includegraphics[width=\PlakatUniversitaetLogoBreite]{./Ressourcen/_Bilder/Universitaet_Logo_RGB.pdf}%
\end{textblock*}%
\begin{center}
\color{black}%
\vspace*{\fill}
{\Huge \bfseries \sffamily \@title }\\[4ex]
{\Large\LehrVeranstaltung}\\[4ex]
{\Large  \@author}\\
\vspace*{\fill}
\vspace*{50mm}
\end{center}}
\makeatother


\newcommand{\MakeTitleNewPage}{\maketitle \newpage \color{black}} % !!! NICHT ENTFERNEN !!!
%%%%%%%%%%%%%%%%%%%%%%%%%%%%%%%%%%%%%%%%%%%%%%%%%%%%%%%%%%%%%%%%%%%%%%%%%%%%%%%%
% TUM-Vorlage: Personenspezifische Informationen
%%%%%%%%%%%%%%%%%%%%%%%%%%%%%%%%%%%%%%%%%%%%%%%%%%%%%%%%%%%%%%%%%%%%%%%%%%%%%%%%
%
% Rechteinhaber:
%     Technische Universität München
%     https://www.tum.de
% 
% Gestaltung:
%     ediundsepp Gestaltungsgesellschaft, München
%     http://www.ediundsepp.de
% 
% Technische Umsetzung:
%     eWorks GmbH, Frankfurt am Main
%     http://www.eworks.de
%
%%%%%%%%%%%%%%%%%%%%%%%%%%%%%%%%%%%%%%%%%%%%%%%%%%%%%%%%%%%%%%%%%%%%%%%%%%%%%%%%

% Für die Person anpassen:

\newcommand{\PersonTitel}{}
\newcommand{\PersonVorname}{Valentin}
\newcommand{\PersonNachname}{Herrmann}
\newcommand{\PersonStadt}{@Ort@}
\newcommand{\PersonAdresse}{%
    @Adresse@\\%
    @Plz@~\PersonStadt%
}
\newcommand{\PersonTelefon}{+49 170 7151443}
\newcommand{\PersonFax}{}
\newcommand{\PersonEmail}{valentin.herrmann@tum.de}
\newcommand{\PersonWebseite}{}


\newcommand{\FakultaetAnsprechpartner}{Prof. Dr. Dr. Jürgen Richter-Gebert, Dr. Vanessa Landgraf}
\newcommand{\LehrstuhlName}{Lehrstuhl für Geometrie und Visualisierung}
% Fakultät:
\newcommand{\FakultaetName}{Fakultät für Mathematik}

\newcommand{\EinstellungBankName}{}
\newcommand{\EinstellungBankIBAN}{}
\newcommand{\EinstellungBankBIC}{}
\newcommand{\EinstellungSteuernummer}{}
\newcommand{\EinstellungUmsatzsteuerIdentifikationsnummer}{}

\hyphenation{} % eigene Silbentrennung                    % !!! DATEI ANPASSEN !!!
%%%%%%%%%%%%%%%%%%%%%%%%%%%%%%%%%%%%%%%%%%%%%%%%%%%%%%%%%%%%%%%%%%%%%%%%%%%%%%%%

\usepackage{amsmath}
\newcommand{\PlakatTitel}{Aufgabenblatt 1 (18. Oktober 2021) - Bearbeitung}



%%%%%%%%%%%%%%%%%%%%%%%%%%%%%%%%%%%%%%%%%%%%%%%%%%%%%%%%%%%%%%%%%%%%%%%%%%%%%%%%
%%%%%%%%%%%%%%%%%%%%%%%%%%%%%%%%%%%%%%%%%%%%%%%%%%%%%%%%%%%%%%%%%%%%%%%%%%%%%%%%
% EINSTELLUNGEN
%%%%%%%%%%%%%%%%%%%%%%%%%%%%%%%%%%%%%%%%%%%%%%%%%%%%%%%%%%%%%%%%%%%%%%%%%%%%%%%%

\definecolor{UniversitaetFarbe}{RGB}{0,101,189}
\input{./Ressourcen/_Informationen.tex}




%%%%%%%%%%%%%%%%%%%%%%%%%%%%%%%%%%%%%%%%%%%%%%%%%%%%%%%%%%%%%%%%%%%%%%%%%%%%%%%%
% DOKUMENT
%%%%%%%%%%%%%%%%%%%%%%%%%%%%%%%%%%%%%%%%%%%%%%%%%%%%%%%%%%%%%%%%%%%%%%%%%%%%%%%%


\setlength{\multicolsep}{0pt}



\usepackage[absolute]{textpos} % Positionierung
\textblockorigin{\PlakatSeiteRand}{\PlakatSeiteRand} % Ursprung für Positionierung


\usepackage{needspace}



\newcommand{\PlakatKopfzeileMitFakultaetslogoVeranstaltungTitel}{%
    \fancyhead[L]{%
        \begin{textblock*}{\PlakatFakultaetLogoBreite}%
                [1,1](\PlakatFakultaetslogoPositionKorrekturLinks,
                    \PlakatKopfzeilePositionUnten + \PlakatFakultaetslogoPositionKorrekturOben)% 
            \includegraphics[width=\PlakatFakultaetLogoBreite]{./Ressourcen/_Bilder/Fakultaet_Logo_RGB.pdf}%
        \end{textblock*}%
        \begin{textblock*}{\textwidth - \PlakatFakultaetLogoBreite - \PlakatFakultaetslogoAbstandRechts}%
                [0,1](\PlakatPositionLinks + \PlakatFakultaetLogoBreite + \PlakatFakultaetslogoAbstandRechts, \PlakatKopfzeilePositionUnten)%
            \color{UniversitaetFarbe}%
            \PlakatKopfzeileSchriftgroesse\selectfont%
            \LehrVeranstaltung\\%
            \theauthor\\%
            \thetitle\\%
            \color{black}%
        \end{textblock*}%
    }
}



\newcommand{\PlakatFusszeileLeer}{
    \fancyfoot[L]{}
}


% Kopf- und Fusszeilen:
\usepackage{fancyhdr}
\renewcommand{\headrulewidth}{0pt} % Linie der Kopfzeile ausblenden
\renewcommand{\footrulewidth}{0pt} % Linie der Fußzeile ausblenden


% Dokument:
\begin{document}

\pagestyle{fancy}
\fancyhf{} % Kopf- und Fusszeile zurücksetzen

% Header (rechts)
\fancyhead[R]{%
    \begin{textblock*}{\PlakatUniversitaetLogoBreite}[1,1]%
            (\paperwidth - \PlakatSeiteRand - \PlakatSeiteRand + \PlakatUniversitaetslogoPositionKorrekturLinks, \PlakatKopfzeilePositionUnten)%
        \includegraphics[width=\PlakatUniversitaetLogoBreite]{./Ressourcen/_Bilder/Universitaet_Logo_RGB.pdf}%
    \end{textblock*}%
}


 % !!! NICHT ENTFERNEN !!!
%%%%%%%%%%%%%%%%%%%%%%%%%%%%%%%%%%%%%%%%%%%%%%%%%%%%%%%%%%%%%%%%%%%%%%%%%%%%%%%%

\title{Bearbeitung zu Aufgabenblatt 1 (gestellt~am~18.10.2021)}
\author{Hausaufgaben Team 02 (Valentin~Herrmann,~Dominique~Wittmann)}

\PlakatKopfzeileMitFakultaetslogoVeranstaltungTitel
\MakeTitleNewPage


\subsection*{Aufgabe 5}
\subsubsection*{1.)}
Bei einem Bruch wird der Zähler durch den Nenner geteilt. Hierbei steht der Zähler über 
und der Nenner unter dem Bruchstrich. Der Zähler ist Element der ganzen Zahlen, der Nenner ist Element der ganzen Zahlen außer 0.\\
Als \emph{gleichnamig} werden zwei Brüche mit gleichem Nenner bezeichnet.\\
Beim \emph{Erweitern} werden Zähler und Nenner eines Bruchs mit dem gleichen
Faktor multipliziert. Der Gesamtwert des Bruchs ändert sich hierdurch nicht.
Dies gilt analog für das \emph{Kürzen}, nur dass hier durch den gleichen Divisor geteilt wird.
Für das vollständige Kürzen wird als Divisor der größter gemeinsame Teiler verwendet.

\subsubsection*{2.)}
\begin{itemize}
    \item[a)] Ein (vollständig) gekürzter Bruch ist ein Bruch, bei dem Zähler und Nenner keinen gemeinsamen Teiler habe. Sie sind also teilerfremd. 
    \item[b)] Um zwei Brüche gleichnamig zu machen, erweitert man die beiden jeweils mit dem Nenner des anderen.
    \item[c)] Zur Multiplikation zweier Brüche multipliziert man den Zähler des einen mit dem Zähler des anderen und analog den Nenner.
    \item[d)] Zur Addition zweier Brüche macht man diese zunächst gleichnamig und addiert anschließend die beiden Zähler zueinander und übernimmt den gemeinsamen Nenner. Ggf. kann das Ergebnis dann noch gekürzt werden.
\end{itemize}

\subsubsection*{3.)}

\begin{align*}
        & a,b \in , a \neq 0, b \neq 0 \\
        \Leftrightarrow & \frac{1}{a}+\frac{1}{b}-\frac{1}{a+b} \\
        \Leftrightarrow & \frac{a+b}{ab}-\frac{1}{a+b} \\
        \Leftrightarrow & \frac{(a+b)^2-ab}{ab(a+b)} \\
        \Leftrightarrow & \frac{(a+b)^2-ab}{a^2b+ab^2}
\end{align*}


\subsection*{Aufgabe 6}
\subparagraph{1.)}
Nein, denn eine Person, die als Maler sehr bekannt ist ("der bekannteste Maler") und auch gleichzeitig dichtet,
ist dann der bekannteste Maler unter den Dichtern. Da diese Person als Dichter nicht unbedingt sehr 
bekannt sein muss, sondern ein anderer Maler ein bekannterer Dichter sein kann, ist der bekannteste
Maler unter den Dichtern nicht gleich der bekannteste Dichter unter den Malern.
\subparagraph{2.)}
Ja, denn die Maler unter den Dichtern bezeichnen die gleiche Schnittmenge, wie die Dichter
unter den Malern. Somit gibt es hier nur eine einheitliche Gruppe, die auch nur eine 
älteste Person hat.
\subparagraph*{3.)}
Ja, denn wie bei 2.) handelt es sich auch hier wieder nur um zwei verschiedene Beschreibungen
derselben Schnittmenge zwischen den beiden Gruppen Maler und Dichter. Somit ist die Reihenfolge
der Eigenschaften auch hier irrelevant.

\subsection*{Aufgabe 7}
\textbf{Gegeben:}\\
$l_k = 0,5$ (Land pro Kuh in Hektar)\\
$l_w = 1$ (Land pro Weizen in Hektar)\\
$l = 20$ (insgesamt verfügbares Land in Hektar)\\
\\
$t_k = 200$ (Zeit pro eine Kuh in Stunden) \\
$t_w = 100$ (Zeit pro einen Hektar Weizen in Stunden) \\ 
$t = 2400$ (insagesamt verfügbare Zeit in Stunden) \\
\\
$G_k = 350$ (Gewinn pro Kuh in Euro)\\
$G_w = 260$ (Gewinn pro Hektar Weizen in Euro)\\
\\
\textbf{Gesucht:}\\
$G_{max} = k \cdot G_k + w \cdot G_w = 350k + 260w$ (größtmöglicher Gesamtgewinn in Abhängigkeit von der Menge an Kühen und Weizen)\\

Nach k auflösen:
\begin{align*}
    l &= k \cdot l_k + w \cdot l_w\\
    \Leftrightarrow 20 &= 0,5k + w &\textbar -w\\
    \Leftrightarrow 0,5k &= 20 - w &\textbar \cdot2\\
    \Leftrightarrow k &= 40-2w
\end{align*}

\begin{align*}
    t &= k \cdot t_k + w \cdot t_w\\
    \Leftrightarrow 2400 &= 200k + 100w &\textbar -100w\\
    \Leftrightarrow 200k &= 2400 - 100w &\textbar \div 200\\
    \Leftrightarrow k &= 12 - 0,5w
\end{align*}

Gleichsetzen:
\begin{align*}
    40-2w &= 12 - 0,5w &\textbar +2w-12\\
    \Leftrightarrow 1,5w &= 38 &\textbar \div 1,5\\
    \Leftrightarrow w &= \frac{56}{3}
\end{align*}

Einsetzen:
\begin{align*}
    k &= 12 - 0,5w\\
    \Leftrightarrow k &= 12 - 0,5 \cdot \frac{56}{3}\\
    \Leftrightarrow k &= \frac{8}{3}
\end{align*}

Teilkühe nicht möglich, daher auf ganze Kühe runden.\\

\textbf{Bei Aufrunden:}
\begin{align*}
    k_{auf} &= 3\\
    k_{auf} &= 40-2w_{auf}\\
    \Leftrightarrow 3 &= 40-2w_{auf} &\textbar -40\\
    \Leftrightarrow -2w_{auf} &= -37 &\textbar \div (-2)\\
    \Leftrightarrow w_{auf} &= 18,5
\end{align*}
Prüfen:
\begin{align*}
    3 \cdot 0,5 + 18,5 = 20 (Land:\surd)\\
    3 \cdot 200 + 18,5 \cdot 100 = 2450 (Zeit:\times)
\end{align*}

50 Stunden Zeit zu viel benötigt, daher einen halben Hektar Weizen (=50h) weniger. Also nur 18 Hektar Weizen ($w_{auf}=18$)\\
$\Rightarrow G_{auf} = 3 \cdot 350 + 18 \cdot 260 = 5730$ [€]\\


\textbf{Bei Abrunden:}
\begin{align*}
    k_{ab} &= 2\\
    k_{ab} &= 40-2w_{ab}\\
    \Leftrightarrow 2 &= 40-2w_{ab} &\textbar -40\\
    \Leftrightarrow -2w_{ab} &= -38 &\textbar \div (-2)\\
    \Leftrightarrow w_{ab} &= 19
\end{align*}
Prüfen:
\begin{align*}
    2 \cdot 0,5 + 19 = 20 (Land:\surd)\\
    2 \cdot 200 + 19 \cdot 100 = 2300 (Zeit:\surd)
\end{align*}

$\Rightarrow G_{ab} = 2 \cdot 350 + 19 \cdot 260 = 5640$ [€]\\

\textbf{Also:}\\
$G_{ab} < G_{auf} \Rightarrow G_{max} = G_{auf} = 5730$ [€]\\
Der Agrarökonom kann also mit 3 Kühen und 18 Hektar Weizen den höchsten Gewinn von 5730€ erzielen.



%%%%%%%%%%%%%%%%%%%%%%%%%%%%%%%%%%%%%%%%%%%%%%%%%%%%%%%%%%%%%%%%%%%%%%%%%%%%%%%%
\end{document} % !!! NICHT ENTFERNEN !!!
%%%%%%%%%%%%%%%%%%%%%%%%%%%%%%%%%%%%%%%%%%%%%%%%%%%%%%%%%%%%%%%%%%%%%%%%%%%%%%%%

